\documentclass[a4paper,12px]{article}

\usepackage{graphicx}
\usepackage[english]{babel}
\usepackage{fancyhdr}
\usepackage{lastpage}
\usepackage{xifthen}
\usepackage[linesnumberedhidden, titlenotnumbered]{algorithm2e}
\usepackage{lipsum}
\usepackage{hyperref}
\usepackage{array}
\usepackage{tabularx}

\usepackage{minted}
\usepackage{caption}
\usepackage{amssymb}

\pagestyle{fancy}
\lhead{\includegraphics[width=7cm]{logoUvA}}
\rhead{\footnotesize \textsc {Paper\\ \titel}}
\lfoot
{
    \footnotesize \studentA
    \ifthenelse{\isundefined{\studentB}}{}{\\ \studentB}
    \ifthenelse{\isundefined{\studentC}}{}{\\ \studentC}
    \ifthenelse{\isundefined{\studentD}}{}{\\ \studentD}
    \ifthenelse{\isundefined{\studentE}}{}{\\ \studentE}
}
\cfoot{}
\rfoot{\small \textsc {Page \thepage\ of \pageref{LastPage}}}
\renewcommand{\footrulewidth}{0.5pt}

\fancypagestyle{firststyle}
{
    \vspace*{6cm}
    \fancyhf{}
    \renewcommand{\headrulewidth}{0pt}
    \chead{\includegraphics[width=7cm]{logoUvA}}
    %\rfoot{\small \textsc {Page \thepage\ of \pageref{LastPage}}}
}

\setlength{\topmargin}{-0.3in}
\setlength{\textheight}{630pt}
\setlength{\headsep}{40pt}

% =================================== DOC INFO ===================================

\newcommand{\titel}{Trade-offs in distributed file systems}
\newcommand{\opdracht}{Examples and applications}
\newcommand{\docent}{Ana Varbanescu}
\newcommand{\cursus}{Concurrency and Parallel Programming}
\newcommand{\vakcode}{5062COPP6Y}
\newcommand{\datum}{\today}
\newcommand{\studentA}{Robin Klusman}
\newcommand{\uvanetidA}{10675671}
\newcommand{\studentB}{Maico Timmerman}
\newcommand{\uvanetidB}{10542590}
%\newcommand{\studentC}{Boudewijn Braams}
\newcommand{\uvanetidC}{10401040}
%\newcommand{\studentD}{Govert Verkes}
\newcommand{\uvanetidD}{10211748}
%\newcommand{\studentE}{Naam student 5}
\newcommand{\uvanetidE}{UvAnetID student 5}

% ===================================  ===================================

\begin{document}
\thispagestyle{firststyle}
\begin{center}
    \rule{\linewidth}{0.5pt} \\[0.4cm]
    {\LARGE \bfseries \titel}
    \rule{\linewidth}{0.5pt} \\[0.2cm]
    \textsc{\Large \opdracht}\\[0.2cm]
    \vspace{0.2cm}
    {\large \datum  \\[0.4cm]}

    \begin{minipage}{0.4\textwidth}
        \begin{flushleft}
            \emph{Authors:}\\
            {\studentA \\ {\small \uvanetidA \\[0.2cm]}}
            \ifthenelse{\isundefined{\studentB}}{}{\studentB \\ {\small \uvanetidB \\[0.2cm]}}
            \ifthenelse{\isundefined{\studentC}}{}{\studentC \\ {\small \uvanetidC \\[0.2cm]}}
            \ifthenelse{\isundefined{\studentD}}{}{\studentD \\ {\small \uvanetidD \\[0.2cm]}}
            \ifthenelse{\isundefined{\studentE}}{}{\studentE \\ {\small \uvanetidE \\ [0.2cm]}}
        \end{flushleft}
    \end{minipage}
    ~
    \begin{minipage}{0.4\textwidth}
        \begin{flushright}
            \emph{Supervisor:} \\
            \docent \\[0.2cm]
            \emph{Course:} \\
            \cursus \\[0.2cm]
            \emph{Course code:} \\
            \vakcode \\[0.2cm]
        \end{flushright}
    \end{minipage}\\[1 cm]
\end{center}


% =================================== FRONT PAGE ===================================

\clearpage
\vspace*{0.5cm}
\tableofcontents
\clearpage



% =================================== MAIN TEXT ===================================

\section{Abstract}

A distributed file system is a client and server system where clients can access
data located on the servers. The servers can be spread over multiple locations
and should still provide reliable access to files. To do so there are a few
requirements that need to be met: universality, performance and fault tolerance.
In this paper our aim is to examine what trade-offs designers of distributed
file systems made in order to make their file systems work. To do this we will
discuss three example distributed file systems, LOCUS, Andrew and Ceph and look
at how they met these requirements or whether they needed to compromise in some
way. We will also compare the three to see how distributed file systems
developed over the years.

\section{Introduction}

A file system is the part of a computer system that manages long term storage of
data.  The most common file system is a centralised file system. This type of
file system is the one we usually find in a personal computer, here files are
stored on a single storage device at a single location.\\ Another type of file
system is the distributed file system, in these file systems data can be stored
on multiple storage devices that also do not necessarily need to be at the same
location. These file systems are useful when multiple people need to share their
data quickly and securely.\\ In this paper our aim is to find out what the
trade-offs are exactly  when choosing a distributed file system over a
centralised file system. And also how distributed file systems developed over
the years. We will do this by looking at three examples and comparing the
properties of those to each other and to a centralised file system.\\

In the next section we take a closer look at what a distributed file system is,
and why it is useful. Then we will discuss in short what the requirements are
for a distributed file system to be effective and useful. In the section
thereafter we will look at a couple of examples of distributed file systems to
see what trade-offs they have had to make. Then the most important trade-offs
will be summarised and discussion in the last section of this paper.

\section{The Distributed File System}

In this section we take a more in depth look at what a distributed file system
exactly is, and why it is useful. A distributed file system is a system of
servers and clients (in some cases devices can be both server and client at the
same time) where data is spread over multiple storage devices (servers) but can
be accessed by clients as though it were on their own device. \cite{concepts} A
distributed file system is particularly convenient in large corporations where
employees each have their own account that they can use to log in to any
computer at the office. The employees will want to access their files regardless
of which computer they are logged in from. A distributed file system can make
sure every employee will have access to their files as though the files were on
the computer currently being used.

\section{Requirements}

Here we will discuss in short what the requirements are of a distributed file
system in order for it to be effective. First of all a user must be able to
access the files they are trying to access in a way that is no different
accessing files on a centralised file system. And a user must also be able to
access their files located on the distributed file system from any machine
within the system. \cite{concepts} Another very important requirement is of
course the performance of the system. If a file system is too slow, it becomes
impractical to use since every time a client requests access to a file they
would have to wait for it. At the same time the file system also needs to be
fault tolerant, this means that the system will not completely crash or shut
down when a fault occurs. Instead it should either manage the fault, or continue
functioning with reduced functionality until the fault is solved.
\cite{concepts}

\section{Distributed File System Examples}
\subsection{LOCUS}


\subsection{Andrew}


\subsection{Ceph}


\section{Discussion}



% =================================== REFERENCES ===================================

\clearpage
\bibliographystyle{unsrt}
\bibliography{bib}

\end{document}

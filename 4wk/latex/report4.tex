\documentclass[a4paper,12px]{article}

\usepackage{graphicx}
\usepackage[english]{babel}
\usepackage{fancyhdr}
\usepackage{lastpage}
\usepackage{xifthen}
\usepackage[linesnumberedhidden, titlenotnumbered]{algorithm2e}
\usepackage{lipsum}
\usepackage{hyperref}
\usepackage{array}
\usepackage{tabularx}

\usepackage{minted}
\usepackage{caption}
\usepackage{amssymb}

\pagestyle{fancy}
\lhead{\includegraphics[width=7cm]{logoUvA}}
\rhead{\footnotesize \textsc {Report\\ \opdracht}}
\lfoot
{
    \footnotesize \studentA
    \ifthenelse{\isundefined{\studentB}}{}{\\ \studentB}
    \ifthenelse{\isundefined{\studentC}}{}{\\ \studentC}
    \ifthenelse{\isundefined{\studentD}}{}{\\ \studentD}
    \ifthenelse{\isundefined{\studentE}}{}{\\ \studentE}
}
\cfoot{}
\rfoot{\small \textsc {Page \thepage\ of \pageref{LastPage}}}
\renewcommand{\footrulewidth}{0.5pt}

\fancypagestyle{firststyle}
{
    \fancyhf{}
    \renewcommand{\headrulewidth}{0pt}
    \chead{\includegraphics[width=7cm]{logoUvA}}
    \rfoot{\small \textsc {Page \thepage\ of \pageref{LastPage}}}
}

\setlength{\topmargin}{-0.3in}
\setlength{\textheight}{630pt}
\setlength{\headsep}{40pt}

% =================================== DOC INFO ===================================

\newcommand{\titel}{MPI}
\newcommand{\opdracht}{Assignment 3.2: Collective Communication}
\newcommand{\docent}{Dr. C. Grelck}
\newcommand{\cursus}{Concurrency and Parallel Programming}
\newcommand{\vakcode}{5062COPP6Y}
\newcommand{\datum}{\today}
\newcommand{\studentA}{Robin Klusman}
\newcommand{\uvanetidA}{10675671}
\newcommand{\studentB}{Maico Timmerman}
\newcommand{\uvanetidB}{10542590}
%\newcommand{\studentC}{Boudewijn Braams}
\newcommand{\uvanetidC}{10401040}
%\newcommand{\studentD}{Govert Verkes}
\newcommand{\uvanetidD}{10211748}
%\newcommand{\studentE}{Naam student 5}
\newcommand{\uvanetidE}{UvAnetID student 5}

% ===================================  ===================================

\begin{document}
\thispagestyle{firststyle}
\begin{center}
    \textsc{\Large \opdracht}\\[0.2cm]
    \rule{\linewidth}{0.5pt} \\[0.4cm]
    {\huge \bfseries \titel}
    \rule{\linewidth}{0.5pt} \\[0.2cm]
    {\large \datum  \\[0.4cm]}

    \begin{minipage}{0.4\textwidth}
        \begin{flushleft}
            \emph{Student:}\\
            {\studentA \\ {\small \uvanetidA \\[0.2cm]}}
            \ifthenelse{\isundefined{\studentB}}{}{\studentB \\ {\small \uvanetidB \\[0.2cm]}}
            \ifthenelse{\isundefined{\studentC}}{}{\studentC \\ {\small \uvanetidC \\[0.2cm]}}
            \ifthenelse{\isundefined{\studentD}}{}{\studentD \\ {\small \uvanetidD \\[0.2cm]}}
            \ifthenelse{\isundefined{\studentE}}{}{\studentE \\ {\small \uvanetidE \\ [0.2cm]}}
        \end{flushleft}
    \end{minipage}
    ~
    \begin{minipage}{0.4\textwidth}
        \begin{flushright}
            \emph{Supervisor:} \\
            \docent \\[0.2cm]
            \emph{Course:} \\
            \cursus \\[0.2cm]
            \emph{Course code:} \\
            \vakcode \\[0.2cm]
        \end{flushright}
    \end{minipage}\\[1 cm]
\end{center}


% =================================== FRONT PAGE ===================================

\vspace{2cm}
\begin{center}
    \includegraphics[width=(\textwidth/5*3)]{parallel_tasks}
\end{center}
\clearpage

\tableofcontents
\vspace{5mm}

% =================================== MAIN TEXT ===================================

\section{Introduction}

In this program MapReduce (Hadoop) is used to process large quantities of data.
The data that will be processed in this case are tweets from 2009. These tweets
will first be scanned for any hashtags to determine which hashtags were the most
popular ones in that specific set of tweets. Then the tweets will be scanned
once again but this time the aim is to determine the language so that the
sentiment value can be determined in case the tweet is in English. Later the
mean and standard deviation of the sentiment values will also be calculated.

Hadoop is as stated above an implementation of MapReduce.

\section{Method}

To accomplish these two tasks, the decision was made to split the program into
two separate programs. One will find all the hashtags used and count them, the
other will calculate the sentiment values.

The first program maps single lines of text to the different nodes. For each
line the program first determines if the line is the actual tweet, if not we do
not process it. Then each line will be checked to see if it contains any
hashtags and if it does, these hashtags will be returned to the reduce function as
a key-value pair where the key is the hashtag and the value is always one (in
case a hashtag is found multiple times in one line, it will just return multiple
pairs, counting is done later). The reduce function will count the total amount
of occurrences of each hashtag and output them (unsorted). This output is then
piped to the `sort' program to sort them and find the most frequent hashtags.

The second part or more accurately, the second program also maps individual
lines of text from the file while discarding those that are not the actual
tweets. When a line is determined to be a tweet, the program determines whether
or not there is any hashtag in that tweet, if not we do not process it any
further. If it however does contain a hashtag the language of that tweet is
determined. If the language then turns out to be English, finally the sentiment
value can be calculated. This value is then returned together with the hashtag
found in the tweet as a key-value pair to the reduce function. In the reduce
function the mean and standard deviation is calculated and outputted.

\section{Results}

\noindent\begin{tabularx}{\textwidth}{p{.22\textwidth} p{.15\textwidth} p{.25\textwidth} p{.25\textwidth}}
    Hashtag & Occurences & Average Sentiment & Standard Deviation \\
    \hline
    \#iranelection  & 32 &  &  \\
    \#honduras      & 14 &  &  \\
    \#forasarney    & 12 &  &  \\
    \#tcot          & 10 &  &  \\
    \#xixicoco      & 9  &  &  \\
    \#xixicoc       & 9  &  &  \\
    \#neda          & 9  &  &  \\
    \#followadd     & 9  &  &  \\
    \#140mafia      & 9  &  &  \\
    \#spymaster     & 8  &  &  \\
\end{tabularx}

\section{Conclusion}


% =================================== REFERENCES ===================================

%\clearpage
%\bibliographystyle{unsrt}
%\bibliography{bib}

\end{document}

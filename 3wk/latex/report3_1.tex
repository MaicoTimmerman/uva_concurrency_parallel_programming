\documentclass[a4paper,12px]{article}

\usepackage{graphicx}
\usepackage[english]{babel}
\usepackage{fancyhdr}
\usepackage{lastpage}
\usepackage{xifthen}
\usepackage[linesnumberedhidden, titlenotnumbered]{algorithm2e}
\usepackage{lipsum}
\usepackage{hyperref}
\usepackage{array}
\usepackage{tabularx}

\usepackage{minted}
\usepackage{caption}
\usepackage{amssymb}

\pagestyle{fancy}
\lhead{\includegraphics[width=7cm]{logoUvA}}
\rhead{\footnotesize \textsc {Report\\ \opdracht}}
\lfoot
{
    \footnotesize \studentA
    \ifthenelse{\isundefined{\studentB}}{}{\\ \studentB}
    \ifthenelse{\isundefined{\studentC}}{}{\\ \studentC}
    \ifthenelse{\isundefined{\studentD}}{}{\\ \studentD}
    \ifthenelse{\isundefined{\studentE}}{}{\\ \studentE}
}
\cfoot{}
\rfoot{\small \textsc {Page \thepage\ of \pageref{LastPage}}}
\renewcommand{\footrulewidth}{0.5pt}

\fancypagestyle{firststyle}
{
    \fancyhf{}
    \renewcommand{\headrulewidth}{0pt}
    \chead{\includegraphics[width=7cm]{logoUvA}}
    \rfoot{\small \textsc {Page \thepage\ of \pageref{LastPage}}}
}

\setlength{\topmargin}{-0.3in}
\setlength{\textheight}{630pt}
\setlength{\headsep}{40pt}

% =================================== DOC INFO ===================================

\newcommand{\titel}{MPI}
\newcommand{\opdracht}{Assignment 3.1: Wave}
\newcommand{\docent}{Dr. C. Grelck}
\newcommand{\cursus}{Concurrency and Parallel Programming}
\newcommand{\vakcode}{5062COPP6Y}
\newcommand{\datum}{\today}
\newcommand{\studentA}{Robin Klusman}
\newcommand{\uvanetidA}{10675671}
\newcommand{\studentB}{Maico Timmerman}
\newcommand{\uvanetidB}{10542590}
%\newcommand{\studentC}{Boudewijn Braams}
\newcommand{\uvanetidC}{10401040}
%\newcommand{\studentD}{Govert Verkes}
\newcommand{\uvanetidD}{10211748}
%\newcommand{\studentE}{Naam student 5}
\newcommand{\uvanetidE}{UvAnetID student 5}

% ===================================  ===================================

\begin{document}
\thispagestyle{firststyle}
\begin{center}
    \textsc{\Large \opdracht}\\[0.2cm]
    \rule{\linewidth}{0.5pt} \\[0.4cm]
    {\huge \bfseries \titel}
    \rule{\linewidth}{0.5pt} \\[0.2cm]
    {\large \datum  \\[0.4cm]}

    \begin{minipage}{0.4\textwidth}
        \begin{flushleft}
            \emph{Student:}\\
            {\studentA \\ {\small \uvanetidA \\[0.2cm]}}
            \ifthenelse{\isundefined{\studentB}}{}{\studentB \\ {\small \uvanetidB \\[0.2cm]}}
            \ifthenelse{\isundefined{\studentC}}{}{\studentC \\ {\small \uvanetidC \\[0.2cm]}}
            \ifthenelse{\isundefined{\studentD}}{}{\studentD \\ {\small \uvanetidD \\[0.2cm]}}
            \ifthenelse{\isundefined{\studentE}}{}{\studentE \\ {\small \uvanetidE \\ [0.2cm]}}
        \end{flushleft}
    \end{minipage}
    ~
    \begin{minipage}{0.4\textwidth}
        \begin{flushright}
            \emph{Supervisor:} \\
            \docent \\[0.2cm]
            \emph{Course:} \\
            \cursus \\[0.2cm]
            \emph{Course code:} \\
            \vakcode \\[0.2cm]
        \end{flushright}
    \end{minipage}\\[1 cm]
\end{center}


% =================================== FRONT PAGE ===================================

\vspace{2cm}
\begin{center}
    \includegraphics[width=(\textwidth/5*3)]{parallel_tasks}
\end{center}
\clearpage

\tableofcontents
\vspace{5mm}

% =================================== MAIN TEXT ===================================

\section{Introduction}

In this assignment we yet again take a look at the wave equation simulation
program. This time it will be parallelised using MPI. This means that message
passing has to be used to communicate between different instances of the program
instead of just being able to write to the same memory space, because each node
has its own memory and accessing another node's memory is relatively slow. So
the transfer of data is kept to a minimum, and we will use messages to
synchronise and exchange small quantities of data.

\section{Method}

The OpenMPI predefined functions are used to simplify the creation of this
program. These functions define actions such as sending a message, receiving a
message, writing to a file in a certain order, etc. Our implementation will make
every instance compute a certain portion of the wave, and then communicate its
edge values to its neighbours, since those neighbours need those edge values to
compute their own edges. After a certain amount of iterations have been
completed, all instances of the program will write to a file in the correct
order to get the full state of the wave.

\section{Results}


\section{Discussion}


% =================================== REFERENCES ===================================

%\clearpage
%\bibliographystyle{unsrt}
%\bibliography{bib}

\end{document}

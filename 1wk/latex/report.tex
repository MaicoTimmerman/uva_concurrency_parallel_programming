\documentclass[a4paper,12px]{article}
\usepackage{graphicx}
\usepackage[english]{babel}
\usepackage{fancyhdr}
\usepackage{lastpage}
\usepackage{xifthen}
\usepackage[linesnumberedhidden, titlenotnumbered]{algorithm2e}
\usepackage{lipsum}
\usepackage{hyperref}
\usepackage{array}
\usepackage{tabularx}

\usepackage{minted}
\usepackage{caption}
\usepackage{amssymb}

\pagestyle{fancy}
\lhead{\includegraphics[width=7cm]{logoUvA}}
\rhead{\footnotesize \textsc {Report\\ \opdracht}}
\lfoot
{
    \footnotesize \studentA
    \ifthenelse{\isundefined{\studentB}}{}{\\ \studentB}
    \ifthenelse{\isundefined{\studentC}}{}{\\ \studentC}
    \ifthenelse{\isundefined{\studentD}}{}{\\ \studentD}
    \ifthenelse{\isundefined{\studentE}}{}{\\ \studentE}
}
\cfoot{}
\rfoot{\small \textsc {Page \thepage\ of \pageref{LastPage}}}
\renewcommand{\footrulewidth}{0.5pt}

\fancypagestyle{firststyle}
{
    \fancyhf{}
    \renewcommand{\headrulewidth}{0pt}
    \chead{\includegraphics[width=7cm]{logoUvA}}
    \rfoot{\small \textsc {Page \thepage\ of \pageref{LastPage}}}
}

\setlength{\topmargin}{-0.3in}
\setlength{\textheight}{630pt}
\setlength{\headsep}{40pt}

% =================================== DOC INFO ===================================

\newcommand{\titel}{POSIX Threads}
\newcommand{\opdracht}{Assignment 1}
\newcommand{\docent}{Dr. A. Pimentel}
\newcommand{\cursus}{Concurrency and Parallel Programming}
\newcommand{\vakcode}{5062COPP6Y}
\newcommand{\datum}{\today}
\newcommand{\studentA}{Robin Klusman}
\newcommand{\uvanetidA}{10675671}
\newcommand{\studentB}{Maico Timmerman}
\newcommand{\uvanetidB}{10542590}
%\newcommand{\studentC}{Boudewijn Braams}
\newcommand{\uvanetidC}{10401040}
%\newcommand{\studentD}{Govert Verkes}
\newcommand{\uvanetidD}{10211748}
%\newcommand{\studentE}{Naam student 5}
\newcommand{\uvanetidE}{UvAnetID student 5}

% ===================================  ===================================

\begin{document}
\thispagestyle{firststyle}
\begin{center}
    \textsc{\Large \opdracht}\\[0.2cm]
    \rule{\linewidth}{0.5pt} \\[0.4cm]
    {\huge \bfseries \titel}
    \rule{\linewidth}{0.5pt} \\[0.2cm]
    {\large \datum  \\[0.4cm]}

    \begin{minipage}{0.4\textwidth}
        \begin{flushleft}
            \emph{Student:}\\
            {\studentA \\ {\small \uvanetidA \\[0.2cm]}}
            \ifthenelse{\isundefined{\studentB}}{}{\studentB \\ {\small \uvanetidB \\[0.2cm]}}
            \ifthenelse{\isundefined{\studentC}}{}{\studentC \\ {\small \uvanetidC \\[0.2cm]}}
            \ifthenelse{\isundefined{\studentD}}{}{\studentD \\ {\small \uvanetidD \\[0.2cm]}}
            \ifthenelse{\isundefined{\studentE}}{}{\studentE \\ {\small \uvanetidE \\ [0.2cm]}}
        \end{flushleft}
    \end{minipage}
    ~
    \begin{minipage}{0.4\textwidth}
        \begin{flushright}
            \emph{Supervisor:} \\
            \docent \\[0.2cm]
            \emph{Course:} \\
            \cursus \\[0.2cm]
            \emph{Course code:} \\
            \vakcode \\[0.2cm]
        \end{flushright}
    \end{minipage}\\[1 cm]
\end{center}


% =================================== FRONT PAGE ===================================

\tableofcontents
\clearpage

% =================================== MAIN TEXT ===================================

\section{Introduction}

For this assignment a parallel programming solution needs to be implemented for
two problems, a wave equation simulation and the Sieve of Eratosthenes. For the
wave simulation the user can specify the amount of wave amplitude points, the
amount of steps it needs to simulate and the desired amount of threads. The
program then calculates all the wave values until it has done the specified
amount of steps.

For the Sieve of Eratosthenes the program will keep producing prime numbers with
the use of multiple threads, thus utilising parallelism, until the user decides
to send a SIGINT to terminate the program.

\section{Method}
\subsection{Wave Equation Simulation}

First the specified amount of threads need to be created, these threads will
then all start executing the function $calc\_wave$. $calc\_wave$ first checks if
there is an amplitude point in the row $t+1$ that needs calculation. This check
is done in ascending order, using a variable $current\_index$ that keeps track
of which amplitude point was the last one being calculated. $current\_index$ is
mutex locked to check and increment it before starting calculation on that
particular amplitude point, so that no two threads waste their time calculating
the same point. Once the $current\_index$ reaches the last point in the wave,
$i\_max$, threads will wait until all other threads finish their calculations.
When the row is completely finished the $current\_index$ is reset and the rows
are rotated, after which an event is generated telling all threads to restart
their routine.

\subsection{Sieve of Eratosthenes}

To find prime numbers, the Sieve of Eratosthenes algorithm is used. This
algorithm finds primes by filtering out non-primes from a constant flow of
natural numbers, $n \in \mathbb{N}$. These numbers are passed through a series
of filters, will filter for multiples of the first prime that was encountered,
this will always be 2. Any number that is not found to be a multiple of 2 by
this filter will be passed on to the next filter. If there is no next filter
that means that the number we have is a prime, since it has passed through all
filters and was not found to be a multiple of anything. So for this prime, a new
filter will then be created. Whenever a new prime is found, it will be printed
for the user. Passing numbers between different filters is done using a queue,
so that the different threads do not need to wait for each other to finish
filtering. These queues have a set length, so that starvation cannot occur for
filters that are later in the chain. When a queue is full, that thread will
simply wait until it is completely empty before its starts filling it up again.



\section{Results}


\section{Discussion}

% =================================== REFERENCES ===================================

%\clearpage
%\bibliographystyle{unsrt}
%\bibliography{bib}

\end{document}

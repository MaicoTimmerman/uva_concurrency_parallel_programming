\documentclass[a4paper,12px]{article}
\usepackage{graphicx}
\usepackage[english]{babel}
\usepackage{fancyhdr}
\usepackage{lastpage}
\usepackage{xifthen}
\usepackage[linesnumberedhidden, titlenotnumbered]{algorithm2e}
\usepackage{lipsum}
\usepackage{hyperref}
\usepackage{array}
\usepackage{tabularx}

\usepackage{minted}
\usepackage{caption}
\usepackage{amssymb}

\pagestyle{fancy}
\lhead{\includegraphics[width=7cm]{logoUvA}}
\rhead{\footnotesize \textsc {Report\\ \opdracht}}
\lfoot
{
    \footnotesize \studentA
    \ifthenelse{\isundefined{\studentB}}{}{\\ \studentB}
    \ifthenelse{\isundefined{\studentC}}{}{\\ \studentC}
    \ifthenelse{\isundefined{\studentD}}{}{\\ \studentD}
    \ifthenelse{\isundefined{\studentE}}{}{\\ \studentE}
}
\cfoot{}
\rfoot{\small \textsc {Page \thepage\ of \pageref{LastPage}}}
\renewcommand{\footrulewidth}{0.5pt}

\fancypagestyle{firststyle}
{
    \fancyhf{}
    \renewcommand{\headrulewidth}{0pt}
    \chead{\includegraphics[width=7cm]{logoUvA}}
    \rfoot{\small \textsc {Page \thepage\ of \pageref{LastPage}}}
}

\setlength{\topmargin}{-0.3in}
\setlength{\textheight}{630pt}
\setlength{\headsep}{40pt}

% =================================== DOC INFO ===================================

\newcommand{\titel}{POSIX Threads}
\newcommand{\opdracht}{Assignment 1}
\newcommand{\docent}{Dr. A. Pimentel}
\newcommand{\cursus}{Concurrency and Parallel Programming}
\newcommand{\vakcode}{5062COPP6Y}
\newcommand{\datum}{\today}
\newcommand{\studentA}{Robin Klusman}
\newcommand{\uvanetidA}{10675671}
\newcommand{\studentB}{Maico Timmerman}
\newcommand{\uvanetidB}{10542590}
%\newcommand{\studentC}{Boudewijn Braams}
\newcommand{\uvanetidC}{10401040}
%\newcommand{\studentD}{Govert Verkes}
\newcommand{\uvanetidD}{10211748}
%\newcommand{\studentE}{Naam student 5}
\newcommand{\uvanetidE}{UvAnetID student 5}

% ===================================  ===================================

\begin{document}
\thispagestyle{firststyle}
\begin{center}
    \textsc{\Large \opdracht}\\[0.2cm]
    \rule{\linewidth}{0.5pt} \\[0.4cm]
    {\huge \bfseries \titel}
    \rule{\linewidth}{0.5pt} \\[0.2cm]
    {\large \datum  \\[0.4cm]}

    \begin{minipage}{0.4\textwidth}
        \begin{flushleft}
            \emph{Student:}\\
            {\studentA \\ {\small \uvanetidA \\[0.2cm]}}
            \ifthenelse{\isundefined{\studentB}}{}{\studentB \\ {\small \uvanetidB \\[0.2cm]}}
            \ifthenelse{\isundefined{\studentC}}{}{\studentC \\ {\small \uvanetidC \\[0.2cm]}}
            \ifthenelse{\isundefined{\studentD}}{}{\studentD \\ {\small \uvanetidD \\[0.2cm]}}
            \ifthenelse{\isundefined{\studentE}}{}{\studentE \\ {\small \uvanetidE \\ [0.2cm]}}
        \end{flushleft}
    \end{minipage}
    ~
    \begin{minipage}{0.4\textwidth}
        \begin{flushright}
            \emph{Supervisor:} \\
            \docent \\[0.2cm]
            \emph{Course:} \\
            \cursus \\[0.2cm]
            \emph{Course code:} \\
            \vakcode \\[0.2cm]
        \end{flushright}
    \end{minipage}\\[1 cm]
\end{center}


% =================================== FRONT PAGE ===================================

\vspace{2cm}
\begin{center}
    \includegraphics[width=(\textwidth/5*3)]{parallel_tasks}
\end{center}
\clearpage

\tableofcontents
\vspace{5mm}

% =================================== MAIN TEXT ===================================

\section{Introduction}

For this assignment a parallel programming solution needs to be implemented for
two problems, a wave equation simulation and the Sieve of Eratosthenes. For the
wave simulation the user can specify the amount of wave amplitude points, the
amount of steps it needs to simulate and the desired amount of threads. The
program then calculates all the wave values until it has done the specified
amount of steps.\\
For the Sieve of Eratosthenes assignment the program will use the Sieve of
Eratosthenes algorithm to continuously produce prime numbers until a certain
amount of primes have been found. It will do this using multiple threads, thus
utilising parallelism.

\section{Method}
\subsection{Wave Equation Simulation}

First the specified amount of threads need to be created, these threads will
then all start executing their routine. This routine gets a certain range of
amplitude values that it is allowed to calculate, this range is determined by
dividing the amount of amplitude points by the amount of threads, so that each
thread has the same workload. When a thread finishes its calculations it locks
and decrements the variable that is keeping track of the amount of busy threads.
The lock makes sure no two threads try to decrement it at the same time, causing
it to never reach 0.  If this variable reaches 0, the thread that made it 0 will
perform the buffer swap and broadcast a signal to all waiting threads that they
can continue their work. When the specified amount of time steps has been
completed, the threads will all terminate and the final wave is returned.

\subsection{Sieve of Eratosthenes}

To find prime numbers, the Sieve of Eratosthenes algorithm is used. This
multi-threaded program uses this algorithm to find primes by filtering out
non-primes from a constant flow of natural numbers, $n \in \mathbb{N}$, that is
generated by the generator thread. In this implementation we actually opted to
skip every even number, since those are all multiples of 2 and would thus be
filtered out anyway. This will improve the performance of the program a little
because only half as many numbers need to be processed by the first filter and
one less filter is needed. The generated numbers are put into a queue of a fixed
size, from which the next filter will grab its input. If a number reached the
end of the filter chain, that means it has to be a prime, since its not a
multiple of any previously encountered number. This number will then be printed,
and a new filter thread will be created to filter for multiples of this newly
found prime.\\
The queues used to pass numbers between different threads are of a set size, so
that no starvation can occur for threads further down the chain. Reading and
writing to and from the queues is done mutually exclusive, so that no race
conditions can exist. A thread will first lock the input buffer to see if there
is anything in there. If there is, it will grab the first value and unlock the
input. Then the value is processed and if it is not filtered out, the output
buffer is locked to write this value to it. If there is no space in the output
buffer, or if the input buffer is empty, the thread will wait for a signal
before it can continue.

\section{Results}
\subsection{Wave Equation Simulation}
Running the Wave Simulation on the DAS4 machine with different values for the
total amount of amplitude points and also varying the amount of threads used,
but keeping the amount of time steps the same at 10000. In the graph below the
results are shown, dividing the time it took to run that particular simulation
on one thread by the time it took to run it on two, four, six and eight threads.

\begin{figure}[H]
    \centering
    \includegraphics[width=\textwidth]{chart1_1}
    \caption{Speed-up factor of Wave Simulations with different amount of total
        amplitude points. Comparing multiple thread to single thread runtime.
        Running a total of 10000 time steps.}
\end{figure}

\subsection{Sieve of Eratosthenes}
On the DAS4 machine the program was unfortunately hard-capped at a certain
number of threads, thus making it it impossible to generate more than 1000
primes. The results presented will thus only include prime generation durations
of no more than 1000 primes. The graph below presents how the amount of primes
the program generates relates to the total runtime it needs.

\begin{figure}[H]
    \centering
    \includegraphics[width=\textwidth]{chart1_2}
    \caption{Total runtime of prime generator with different amounts of total
        primes to be generated.}
\end{figure}


\section{Discussion}
\subsection{Wave Equation Simulation}
It is interesting to see that when running the Wave Simulation with one million
amplitude points gives a significantly higher speed-up than when it is run with
a hundred thousand or ten million. In fact, when going from six to eight cores
in the ten million points simulation, the speed-up factor actually slightly
drops. This overhead has to be created then by the synchronization at the end of
each time step. Since the program now has to deal with rather large quantities
of data.\\
The reason for the lower amount of amplitude points to also have a lower
speed-up factor can be explained by the fact that running it with one thread is
relatively fast due to the smaller amount of computations needed. Thus the
speed-up achieved is also lower.


\subsection{Sieve of Eratosthenes}
Here it can be seen that increasing the amount of primes to be generated has a
linear relation to how long it takes to generate them. This is not what one
would expect, since double the amount of primes means that much more than double
the amount of numbers need to be checked. This linear growth can however be
explained by realising that each new prime creates a new thread for checking. So
the when generating more primes, the program also becomes increasingly
concurrent. This again helps reduce the total runtime by utilising the large
amount of cores on the DAS4 machine.



% =================================== REFERENCES ===================================

%\clearpage
%\bibliographystyle{unsrt}
%\bibliography{bib}

\end{document}

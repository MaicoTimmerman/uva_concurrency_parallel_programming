\documentclass[a4paper,12px]{article}

\usepackage{graphicx}
\usepackage[english]{babel}
\usepackage{fancyhdr}
\usepackage{lastpage}
\usepackage{xifthen}
\usepackage[linesnumberedhidden, titlenotnumbered]{algorithm2e}
\usepackage{lipsum}
\usepackage{hyperref}
\usepackage{array}
\usepackage{tabularx}

\usepackage{minted}
\usepackage{caption}
\usepackage{amssymb}

\pagestyle{fancy}
\lhead{\includegraphics[width=7cm]{logoUvA}}
\rhead{\footnotesize \textsc {Report\\ \opdracht}}
\lfoot
{
    \footnotesize \studentA
    \ifthenelse{\isundefined{\studentB}}{}{\\ \studentB}
    \ifthenelse{\isundefined{\studentC}}{}{\\ \studentC}
    \ifthenelse{\isundefined{\studentD}}{}{\\ \studentD}
    \ifthenelse{\isundefined{\studentE}}{}{\\ \studentE}
}
\cfoot{}
\rfoot{\small \textsc {Page \thepage\ of \pageref{LastPage}}}
\renewcommand{\footrulewidth}{0.5pt}

\fancypagestyle{firststyle}
{
    \fancyhf{}
    \renewcommand{\headrulewidth}{0pt}
    \chead{\includegraphics[width=7cm]{logoUvA}}
    \rfoot{\small \textsc {Page \thepage\ of \pageref{LastPage}}}
}

\setlength{\topmargin}{-0.3in}
\setlength{\textheight}{630pt}
\setlength{\headsep}{40pt}

% =================================== DOC INFO ===================================

\newcommand{\titel}{GPGPU: CUDA}
\newcommand{\opdracht}{Assignment 5.2: Parallel Reduction}
\newcommand{\docent}{Dr. R.G. Belleman}
\newcommand{\cursus}{Concurrency and Parallel Programming}
\newcommand{\vakcode}{5062COPP6Y}
\newcommand{\datum}{\today}
\newcommand{\studentA}{Robin Klusman}
\newcommand{\uvanetidA}{10675671}
\newcommand{\studentB}{Maico Timmerman}
\newcommand{\uvanetidB}{10542590}
%\newcommand{\studentC}{Boudewijn Braams}
\newcommand{\uvanetidC}{10401040}
%\newcommand{\studentD}{Govert Verkes}
\newcommand{\uvanetidD}{10211748}
%\newcommand{\studentE}{Naam student 5}
\newcommand{\uvanetidE}{UvAnetID student 5}

% ===================================  ===================================

\begin{document}
\thispagestyle{firststyle}
\begin{center}
    \textsc{\Large \opdracht}\\[0.2cm]
    \rule{\linewidth}{0.5pt} \\[0.4cm]
    {\huge \bfseries \titel}
    \rule{\linewidth}{0.5pt} \\[0.2cm]
    {\large \datum  \\[0.4cm]}

    \begin{minipage}{0.4\textwidth}
        \begin{flushleft}
            \emph{Student:}\\
            {\studentA \\ {\small \uvanetidA \\[0.2cm]}}
            \ifthenelse{\isundefined{\studentB}}{}{\studentB \\ {\small \uvanetidB \\[0.2cm]}}
            \ifthenelse{\isundefined{\studentC}}{}{\studentC \\ {\small \uvanetidC \\[0.2cm]}}
            \ifthenelse{\isundefined{\studentD}}{}{\studentD \\ {\small \uvanetidD \\[0.2cm]}}
            \ifthenelse{\isundefined{\studentE}}{}{\studentE \\ {\small \uvanetidE \\ [0.2cm]}}
        \end{flushleft}
    \end{minipage}
    ~
    \begin{minipage}{0.4\textwidth}
        \begin{flushright}
            \emph{Supervisor:} \\
            \docent \\[0.2cm]
            \emph{Course:} \\
            \cursus \\[0.2cm]
            \emph{Course code:} \\
            \vakcode \\[0.2cm]
        \end{flushright}
    \end{minipage}\\[1 cm]
\end{center}


% =================================== FRONT PAGE ===================================

\vspace{2cm}
\begin{center}
    \includegraphics[width=(\textwidth/5*3)]{parallel_tasks}
\end{center}
\clearpage

\tableofcontents
\vspace{5mm}

% =================================== MAIN TEXT ===================================

\section{Introduction}

In this assignment the goal is to determine the maximum value in an array of
floats in fast way using CUDA to parallelise the task. Since we are dealing with
a large amount of simple tasks (comparing two values and saving the larger one)
CUDA should have excellent performance when used for this task.

\section{Method}

In the introduction we spoke of comparing two values, this is of course
necessary to determine a maximum value. Our implementation will compare the
value in index $i$ to the value at index $n-i$ (where $n$ is the size of the
array) and write the largest of the two values at index $i$. After completing
a cycle the latter half of the array is no longer relevant, so we will then
perform the same task on the first half of the array. This will continue until
one value is left over which is then guaranteed to be the largest value in the
array.

\section{Results}

The results below illustrate the difference in performance when running the
reduction program with shared memory versus non-shared memory. The performance
with shared memory is as expected better than that of the non-shared memory
counterpart. We also compared the sequential execution of the program using
shared memory with non-shared memory and here it turns out to make no
difference.
Another interesting fact we see in this graph is that the sequential execution
is much faster than the GPGPU execution.

\begin{center}
    \includegraphics[width=\textwidth]{reduce-max}
\end{center}
\captionof{figure}{Comparison of shared memory execution to non-shared memory
execution for both CUDA and Sequential execution of the reduction program
operating on an array of $2*1024^2$ floats.}

\section{Conclusion}

We will start by discussing the more straightforward result that we see in the
graph above; shared memory execution increases performance for the CUDA
implementation. This is easily explained by the overhead created by
reading remote memory locations. Since a lot of data has to be exchanged (we
need both the neighbours' values to calculate the new value for an amplitude
point) the overhead created by accessing remote memory is much larger overall
than the overhead created by having a shared memory system.
For sequential execution this makes no difference since it there is only one
instance running so no remote accesses have to take place, it only needs to
access its own memory. \\
Then there is the interesting results that says that sequential execution is
much faster than parallel execution with CUDA. This result can only be explained
as an error, because it is impossible that sequential execution on this large
quantity of data is faster than parallel execution. Either there has been a
mistake in timing the sequential execution, or something went wrong in writing
this program.



% =================================== REFERENCES ===================================

%\clearpage
%\bibliographystyle{unsrt}
%\bibliography{bib}

\end{document}
